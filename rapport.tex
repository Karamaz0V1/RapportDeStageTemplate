% Rapport de Stage -- École supérieure d'ingénieurs de Rennes
% Florent GUIOTTE
\documentclass[a4paper,12pt,twoside,final]{article}

% --- Packages --- 
\usepackage[english,francais]{babel}
\usepackage[utf8]{inputenc}
\usepackage[T1]{fontenc}
\usepackage[pdftex]{graphicx}
\usepackage{setspace}
\usepackage[french]{varioref}
\usepackage{fancyhdr}
\usepackage{lmodern}
\usepackage{xcolor}
\usepackage{titlesec}
\usepackage{listings}
\usepackage{pdfpages}
\usepackage{subfig}
\usepackage{wrapfig}
\usepackage{numprint}
\usepackage{hyperref}
\usepackage{glossaries}
%\usepackage[left=3cm, right=3cm]{geometry} % Modifier les marges /!\ avec fancyhdr

% --- Includes --- 
\colorlet{maincolor}{blue!30!black!79!white}
\colorlet{glocolor}{blue!50!white}
\colorlet{subgray}{white!20!black}
\colorlet{subsubgray}{white!40!black}

% Report
\newcommand{\reporttitle}{\textsc{Esir} Cloud}     % Titre
\newcommand{\reportsubject}{Rapport de stage ouvrier \\\large École supérieure d'ingénieurs de Rennes} % Sujet
\newcommand{\reportauthor}{Florent \textsc{Guiotte}} % Auteur
\newcommand{\reporteta}{esir}

% University
\newcommand{\univname}{\textsc{Esir}} % Name
\newcommand{\univaddra}{Université de Rennes 1} % Address line 1
\newcommand{\univaddrb}{Campus de Beaulieu \\ \numprint{35042} \textsc{Rennes} Cédex} % Address line 2
\newcommand{\univlogo}{img/logo_esir.pdf} % Logo

% Company
\newcommand{\compname}{\textsc{Istic}} % Name
\newcommand{\compaddra}{\univaddra} % Address line 1
\newcommand{\compaddrb}{\univaddrb} % Adress line 2
\newcommand{\complogo}{img/logo_istic.png} % Logo

% Other
\newcommand{\supervisor}{François \textsc{Dagorn} }
\newcommand{\professor}{Jean-Christophe \textsc{Engel} } 
\newcommand{\datewp}{du 19 juin au 18 juillet 2014 }

% PDF
\newcommand{\pdfkw}{{esir} {istic} {cloud} {stage} {ouvrier} {ldap} {msdos}} % Keywords

\newcommand{\esir}{école supérieur d'ingénieurs de Rennes }
\newcommand{\wdav}{Web\textsc{Dav} }
\newcommand{\rI}{Rennes~1 }
\newcommand{\urI}{Université de \rI }

% TODO : Include pour du code (C, C++, ...)


\newcommand{\HRule}{{\color{maincolor} \rule{\linewidth}{0.5mm}}}
\setlength{\parskip}{1ex} % Espace entre les paragraphes
\titleformat*{\section}{\pagestyle{empty}\sffamily\bfseries\color{maincolor}\LARGE}
\titleformat*{\subsection}{\sffamily\bfseries\color{subgray}\large}
\titleformat*{\subsubsection}{\sffamily\bfseries\color{subsubgray}\normalsize}
\fancyhead[LE]{\reportauthor}
\fancyhead[CE]{\reporteta}
\fancyhead[RE]{}
\fancyhead[LO]{}

% --- Glossaire ---
\makeglossaries
\glossarystyle{altlist}
%\glossarystyle{altlistgroup}
\usepackage{xparse}
\DeclareDocumentCommand{\newdualentry}{ O{} O{} m m m m } {
  \newglossaryentry{gls-#3}{name={#5},text={#5\glsadd{#3}},
    description={#6},#1
  }
  \newacronym[see={[Glossary:]{gls-#3}},#2]{#3}{#4}{#5\glsadd{gls-#3}}
}
%\newdualentry{label}{acronym}{version longue acronym}{Def bien longue}

\newglossaryentry{mot}{
    name = mot compliqué,
    description = {ça c'est de la définition},
    plural = mots compliqués
}

\newacronym{esir}{ESIR}{École Supérieure d'ingénieurs de Rennes}



\newcommand{\newempty}{
  \pagestyle{empty}
  \cleardoublepage
}

\newcommand{\newpartie}{
  \pagestyle{empty}
  \cleardoublepage
  \pagestyle{fancy}
}

\newcommand{\newtitle}{
  \pagestyle{empty}
  \cleardoublepage
  \pagestyle{plain}
}

\newcommand{\voidsheet}{
    \pagestyle{empty}
    \cleardoublepage
    \null
    \newpage
    \null
    \newpage
}

\hypersetup{
    pdftitle={\reporttitle},
    pdfauthor={\reportauthor},
    pdfsubject={\reportsubject},
    pdfkeywords=\pdfkw
}

% **************************
% * Agencement des parties *
% **************************

\begin{document}
\renewcommand\contentsname{Sommaire}    % Remplace Table des matières
%\linespread{1.5}
\pagenumbering{alph}
%    % Inspiré de http://en.wikibooks.org/wiki/LaTeX/Title_Creation

\begin{titlepage}

\begin{center}

\begin{minipage}[t]{0.48\textwidth}
  \begin{flushleft}
    \includegraphics [width=30mm]{\univlogo} \\[0.5cm]
    \begin{spacing}{1} % Tu peux changer là, 1.5 ça donne bien
        \LARGE \univname\\
	\Large \univaddra\\
	\large \univaddrb
    \end{spacing}
  \end{flushleft}
\end{minipage}
\begin{minipage}[t]{0.48\textwidth}
  \begin{flushright}
    \includegraphics [width=30mm]{\complogo} \\[0.5cm]
    \begin{spacing}{1}
        \LARGE \compname\\
	\Large \compaddra\\
	\large \compaddrb
    \end{spacing}
  \end{flushright}
\end{minipage} \\[1.5cm]

\textsc {\Large \ttfamily \color{subsubgray}\reportsubject}\\[0.5cm]
\HRule \\[0.7cm]
{\huge \bfseries \sffamily\reporttitle}\\[0.4cm]
\HRule \\[3.0cm]

\vfill

\begin{minipage}[t]{0.6\textwidth}
  \begin{flushleft} \large
    \emph{Responsables :} \\
    \supervisor\\
    \professor
  \end{flushleft}
\end{minipage}
\begin{minipage}[t]{0.3\textwidth}
  \begin{flushright} \large
    \emph{Auteur :}\\
    \reportauthor
  \end{flushright}
\end{minipage}

\vfill

{\large \datewp}

\end{center}

\end{titlepage}

%    \voidsheet
    % Inspiré de http://en.wikibooks.org/wiki/LaTeX/Title_Creation

\begin{titlepage}

\begin{center}

\begin{minipage}[t]{0.48\textwidth}
  \begin{flushleft}
    \includegraphics [width=30mm]{\univlogo} \\[0.5cm]
    \begin{spacing}{1} % Tu peux changer là, 1.5 ça donne bien
        \LARGE \univname\\
	\Large \univaddra\\
	\large \univaddrb
    \end{spacing}
  \end{flushleft}
\end{minipage}
\begin{minipage}[t]{0.48\textwidth}
  \begin{flushright}
    \includegraphics [width=30mm]{\complogo} \\[0.5cm]
    \begin{spacing}{1}
        \LARGE \compname\\
	\Large \compaddra\\
	\large \compaddrb
    \end{spacing}
  \end{flushright}
\end{minipage} \\[1.5cm]

\textsc {\Large \ttfamily \color{subsubgray}\reportsubject}\\[0.5cm]
\HRule \\[0.7cm]
{\huge \bfseries \sffamily\reporttitle}\\[0.4cm]
\HRule \\[3.0cm]

\vfill

\begin{minipage}[t]{0.6\textwidth}
  \begin{flushleft} \large
    \emph{Responsables :} \\
    \supervisor\\
    \professor
  \end{flushleft}
\end{minipage}
\begin{minipage}[t]{0.3\textwidth}
  \begin{flushright} \large
    \emph{Auteur :}\\
    \reportauthor
  \end{flushright}
\end{minipage}

\vfill

{\large \datewp}

\end{center}

\end{titlepage}

    \newempty
%    \section*{Remerciements}

\begin{flushright}

\end{flushright}

% \up{Me} % M\up{Me}
% \supervisor 

%    \newtitle
    \pagenumbering{arabic}
    \tableofcontents
    \newtitle
%    \setlength{\baselineskip}{1.5\baselineskip}
    \section*{Introduction} % Pas de numérotation
\addcontentsline{toc}{section}{Introduction}


    \newpartie
    \section{Partie 1}
\subsection{Sous partie A}
\subsubsection{Sous sous partie a}

% \emph{mettre en evidence}
% \footnote{Note de pied de page}
% \ieme
% \LaTeX
% \label{partie2}
% \ref{fig_img} \pageref{fig_img}
% \textbf{bold}
% \textsc{Guiotte}
% - -- ---

%\begin{figure}[!ht]
%    \center
%    \label{fig_img}
%    \includegraphics[width=0.5\textwidth]{./img/img.jpg}
%    \caption{Légende}
%\end{figure}

%\begin{itemize}
%    \item Élément 1
%    \item Élément 2
%    \item Élément 3
%\end{itemize}

%\begin{figure}[!ht]
%    \center
%    \begin{tabular}{|c||l|r|c|}
%        \hline 
%        Semaine & Tâche importante & Tps estimé & Tps réalisé \\
%        \hline
%        1 & Exercice Sudoku & 35 & 45 \\
%        \hline
%    \end{tabular}
%    \caption{Legende}
%    \label{etiquette}
%\end{figure}

%\begin{figure}[!ht]%htp]
%  \centering
%  \subfloat[Img1]{\label{fig_img1}\includegraphics[width=0.48\textwidth]{img/img1.png}}
%  \hspace{0.030\textwidth}
%  \subfloat[Img2]{\label{fig_img2}\includegraphics[width=0.48\textwidth]{img/img2.png}}
%  \caption{Legende}
%  \label{figs}
%\end{figure}

    \newpartie
    \section{Partie 2}
\subsection{Sous partie A}
\subsubsection{Sous sous partie a}

% \emph{mettre en evidence}
% \footnote{Note de pied de page}
% \ieme
% \LaTeX
% \label{partie2}
% \ref{fig_img} \pageref{fig_img}
% \textbf{bold}
% \textsc{Guiotte}
% - -- ---

%\begin{figure}[!ht]
%    \center
%    \label{fig_img}
%    \includegraphics[width=0.5\textwidth]{./img/img.jpg}
%    \caption{Légende}
%\end{figure}

%\begin{itemize}
%    \item Élément 1
%    \item Élément 2
%    \item Élément 3
%\end{itemize}

%\begin{figure}[!ht]
%    \center
%    \begin{tabular}{|c||l|r|c|}
%        \hline 
%        Semaine & Tâche importante & Tps estimé & Tps réalisé \\
%        \hline
%        1 & Exercice Sudoku & 35 & 45 \\
%        \hline
%    \end{tabular}
%    \caption{Legende}
%    \label{etiquette}
%\end{figure}

%\begin{figure}[!ht]%htp]
%  \centering
%  \subfloat[Img1]{\label{fig_img1}\includegraphics[width=0.48\textwidth]{img/img1.png}}
%  \hspace{0.030\textwidth}
%  \subfloat[Img2]{\label{fig_img2}\includegraphics[width=0.48\textwidth]{img/img2.png}}
%  \caption{Legende}
%  \label{figs}
%\end{figure}

    \newpartie
    \section{Partie 3}
\subsection{Sous partie A}
\subsubsection{Sous sous partie a}

% \emph{mettre en evidence}
% \footnote{Note de pied de page}
% \ieme
% \LaTeX
% \label{partie2}
% \ref{fig_img} \pageref{fig_img}
% \textbf{bold}
% \textsc{Guiotte}
% - -- ---

%\begin{figure}[!ht]
%    \center
%    \label{fig_img}
%    \includegraphics[width=0.5\textwidth]{./img/img.jpg}
%    \caption{Légende}
%\end{figure}

%\begin{itemize}
%    \item Élément 1
%    \item Élément 2
%    \item Élément 3
%\end{itemize}

%\begin{figure}[!ht]
%    \center
%    \begin{tabular}{|c||l|r|c|}
%        \hline 
%        Semaine & Tâche importante & Tps estimé & Tps réalisé \\
%        \hline
%        1 & Exercice Sudoku & 35 & 45 \\
%        \hline
%    \end{tabular}
%    \caption{Legende}
%    \label{etiquette}
%\end{figure}

%\begin{figure}[!ht]%htp]
%  \centering
%  \subfloat[Img1]{\label{fig_img1}\includegraphics[width=0.48\textwidth]{img/img1.png}}
%  \hspace{0.030\textwidth}
%  \subfloat[Img2]{\label{fig_img2}\includegraphics[width=0.48\textwidth]{img/img2.png}}
%  \caption{Legende}
%  \label{figs}
%\end{figure}

    \newtitle
    \phantomsection
\section*{Conclusion}
\addcontentsline{toc}{section}{Conclusion}


    \newtitle
%    \section*{Résumé} % Pas de numérotation
\addcontentsline{toc}{section}{Résumé}

%    \newtitle
%    \section*{Abstract}
\addcontentsline{toc}{section}{Abstract}


%    \newtitle
%    \setlength{\baselineskip}{1.0\baselineskip}
%    % Remplir les définitions dans glossaire.tex
\phantomsection
\addcontentsline{toc}{section}{Glossaire}
\printglossary[title=Glossaire,toctitle=Glossaire et abréviations]

%    \newtitle
%    \phantomsection\addcontentsline{toc}{section}{Références}
\begin{thebibliography}{ABC}	
    \bibitem[REF]{reference} auteur. \emph{titre}. édition, année.
    \bibitem[LPP]{lpp} Rolland. \emph{LaTeX par la pratique}. O'Reilly, 1999.
\end{thebibliography}

%    \newtitle
%    \listoffigures
%    \newtitle
%    \phantomsection
\addcontentsline{toc}{section}{Annexes}
\section*{Annexes} % Pas de numérotation

\appendix
\setcounter{section}{1}
%\setcounter{subsection}{1}

\subsection{Première annexe}
%\includepdf{./annexes/premier.pdf}
%\clearpage

\end{document}
